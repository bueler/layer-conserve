\documentclass[11pt]{amsart}

\usepackage{verbatim}

% math macros
\newcommand\bb{\mathbf{b}}
\newcommand\bbf{\mathbf{f}}
\newcommand\bn{\mathbf{n}}
\newcommand\bq{\mathbf{q}}
\newcommand\bu{\mathbf{u}}
\newcommand\bv{\mathbf{v}}
\newcommand\by{\mathbf{y}}

\newcommand\bQ{\mathbf{Q}}
\newcommand\bV{\mathbf{V}}
\newcommand\bX{\mathbf{X}}

\newcommand\CC{\mathbb{C}}
\newcommand{\DDt}[1]{\ensuremath{\frac{d #1}{d t}}}
\newcommand{\ddt}[1]{\ensuremath{\frac{\partial #1}{\partial t}}}
\newcommand{\ddx}[1]{\ensuremath{\frac{\partial #1}{\partial x}}}
\newcommand{\ddy}[1]{\ensuremath{\frac{\partial #1}{\partial y}}}
\newcommand{\ddxp}[1]{\ensuremath{\frac{\partial #1}{\partial x'}}}
\newcommand{\ddz}[1]{\ensuremath{\frac{\partial #1}{\partial z}}}
\newcommand{\ddxx}[1]{\ensuremath{\frac{\partial^2 #1}{\partial x^2}}}
\newcommand{\ddyy}[1]{\ensuremath{\frac{\partial^2 #1}{\partial y^2}}}
\newcommand{\ddxy}[1]{\ensuremath{\frac{\partial^2 #1}{\partial x \partial y}}}
\newcommand{\ddzz}[1]{\ensuremath{\frac{\partial^2 #1}{\partial z^2}}}
\newcommand{\Div}{\nabla\cdot}
\newcommand\eps{\epsilon}
\newcommand{\grad}{\nabla}
\newcommand{\ihat}{\mathbf{i}}
\newcommand{\ip}[2]{\ensuremath{\left<#1,#2\right>}}
\newcommand{\jhat}{\mathbf{j}}
\newcommand{\khat}{\mathbf{k}}
\newcommand{\nhat}{\mathbf{n}}
\newcommand\lam{\lambda}
\newcommand\lap{\triangle}
\newcommand\Matlab{\textsc{Matlab}\xspace}
\newcommand\RR{\mathbb{R}}
\newcommand\vf{\varphi}


\title{Documentation of code \texttt{layer.c}}

\author{Ed Bueler}


\begin{document}

\begin{abstract}
This brief document explains the code \texttt{layer.c}.  It covers the equations which are solved, and provides some documentation of the runtime options.  It states the way the PDE (strong) form is discretized into a nonlinear system solved by a \texttt{SNESVI} object from PETSc.

This example has been used in my AGU 2014 and LIWG 2015 talks.  Also see the paper \emph{Conservation for fluid layers with free boundaries}, namely the file \verb|../paper/lc.pdf|.
\end{abstract}

\maketitle

\thispagestyle{empty}

\baselineskip=12pt
\parskip=5pt


\section{Introduction}

The code solves a conservation equation for a layer in one dimension,
\begin{equation}
u_t + q_x = f  \label{eq:conserve}
\end{equation}
where $u(t,x)$ is the layer thickness, thus
\begin{equation}
u \ge 0. \label{eq:constraint}
\end{equation}
The source term $f=f(x)$ only depends on space.  The spatial domain is $[0,L]$, where $L=10$, with periodic boundary conditions.  The temporal domain is $[0,T]$.

The flux $q(x,u,u_x)$ is a mix of advective and non-linearly diffusive parts according to a parameter $0\le \lambda \le 1$.  The two parts are $q^0(x,u,u_x)$, a shallow ice approximation flux using a continuously-differentiable bed elevation $b(x)$, and $q^1(u)$, a constant-velocity advection.  In detail,
\begin{align}
    q &= \lambda q^0 + (1-\lambda) q^1,  \label{eq:fluxform} \\
  q^0 &= - \gamma u^{n+2} |(u+b)_x|^{n-1} (u+b)_x, \notag \\
  q^1 &= v_0 u. \notag
\end{align}
Here $n\ge 1$ is the Glen exponent and $\gamma>0$ is a scaling.  We assume the advection is right-ward, so $v_0 \ge 0$.

To get started do this to build and see program-specific options:
\begin{verbatim}
$ make layer
$ ./layer -help |grep lay_
\end{verbatim}
Simple runs that illustrates what the program does is:
\begin{verbatim}
$ ./layer -draw_pause 0.5
$ ./layer -draw_pause 0.1 -lay_dt 0.01 -lay_steps 50 -da_grid_x 200
\end{verbatim}
The second run is at higher space and time resolution.

\section{PDE discretization}

The temporal grid is $t_n = n\Delta t$ where $\Delta t = T/N$ where $N$ defaults to 10 and is set by \verb|-lay_steps|.  In fact $\Delta t$ is set by \verb|-lay_dt|, and $T$ is computed by $T=N\Delta t$.

Let
  $$\Delta x = L / m$$
where $m$ defaults to $50$ and is set by \verb|-da_grid_x| or \verb|-da_refine|.  The spatial grid (i.e.~\emph{regular} grid) points are
    $$x_j = \frac{\Delta x}{2} + j \Delta x$$
for $j=0,\dots,m-1$.  Thus the grid point $x_j$ is centered in $[j\Delta x,(j+1)\Delta x]$.

The basic discretization of \eqref{eq:conserve} is the trapezoid rule in time and centered in space,
\begin{equation}
\frac{u_j^n - u_j^{n-1}}{\Delta t} + \frac{1}{2} \frac{q_{j+1/2}(\bu^n) - q_{j-1/2}(\bu^n)}{\Delta x} + \frac{1}{2} \frac{q_{j+1/2}(\bu^{n-1}) - q_{j-1/2}(\bu^{n-1})}{\Delta x} = f(x_j) \label{eq:basediscret}
\end{equation}
where $\bu^n$ is the vector in $\RR^m$ of all unknowns $u_j^n$, for $j=0,\dots,m-1$, and where $\bu^{n-1}$ is the vector of all $u_j^{n-1}$.  We write the unknowns more simply as
  $$u_j = u_j^n,$$
so $\bu_j = \bu_j^n$.

To do the time-step implicitly we reformulate \eqref{eq:basediscret} as a nonlinear system of $m$ equations in $m$ unknowns $u_j$.  Furthermore we recall the flux split \eqref{eq:fluxform}.  Thus the nonlinear system is
\begin{equation}
F_j(\bu) = 0 \qquad \text{ for } j = 0,\dots,m-1
\end{equation}
where
\begin{align}
F_j(\bu) &= u_j - u_j^{n-1} - \Delta t\,f(x_j)  \label{eq:defineFj} \\
   &\qquad + \lambda \nu_2 \left[q^0_{j+1/2}(\bu) - q^0_{j-1/2}(\bu)\right] + \lambda \nu_2 \left[q^0_{j+1/2}(\bu^{n-1}) - q^0_{j-1/2}(\bu^{n-1})\right] \notag \\
   &\qquad + (1-\lambda) \nu_2 \left[q^1_{j+1/2}(\bu) -  q^1_{j-1/2}(\bu)\right] + (1-\lambda) \nu_2 \left[q^1_{j+1/2}(\bu^{n-1}) - q^1_{j-1/2}(\bu^{n-1})\right] \notag
\end{align}
where
    $$\nu_2 = \frac{\Delta t}{2\Delta x}.$$

\section{Flux discretization}

The flux-discretization scheme for the parts inside square brackets in \eqref{eq:defineFj} determines many details in the code.  These approximations appear in \verb|FormFunctionLocal()|, by call-back from SNES using DMDA information.

Recall flux form \eqref{eq:fluxform}.  We start with approximations of $q^0$.  Let
	$$S(u,z) = - \gamma u^{n+2} |z|^{n-1} z,$$
so that $q^0(x,u,u_x) = S(u,u_x+b_x)$.  Also define derivatives of the bed
	$$b'_{j+1/2} = \frac{b(x_{j+1}) - b(x_j)}{\Delta x}, \quad b'_{j-1/2} = \frac{b(x_j) - b(x_{j-1})}{\Delta x}.$$
Then the diffusive fluxes appearing in \eqref{eq:defineFj}, at time $t_n$, i.e.~using unknown thicknesses $\bu$, are
    $$q^0_{j+1/2}(\bu) = S\left(\frac{u_j+u_{j+1}}{2},\frac{u_{j+1}-u_j}{\Delta x} + b'_{j+1/2}\right),$$
    $$q^0_{j-1/2}(\bu) = S\left(\frac{u_{j-1}+u_j}{2},\frac{u_j-u_{j-1}}{\Delta x} + b'_{j-1/2}\right),$$
The ``old'' diffusive fluxes $q^0_{j+1/2}(\bu^{n-1})$ and $q^0_{j-1/2}(\bu^{n-1})$ in \eqref{eq:defineFj} are defined the same way.

The discretization of the advective flux differences in \eqref{eq:defineFj}, i.e.~the differences of $q^1$ inside square brackets, is user-chooseable according to the option
\begin{quote}
\verb|-lay_adscheme X|
\end{quote}
where \verb|X| is 0,1,2:
\medskip
\renewcommand{\labelenumi}{\texttt{X}$=$\arabic{enumi}: \quad}
\begin{enumerate}
\setcounter{enumi}{-1}
\item %0
A backward-Euler time-discretization and a centered (second-order) space-discretization, so the first line approximates ``$2 \Delta x\,(v_0 u)_x$'':
\begin{align*}
q^1_{j+1/2}(\bu) - q^1_{j-1/2}(\bu)             &= v_0 \left( - u_{j-1} + u_{j+1} \right), \\
q^1_{j+1/2}(\bu^{n-1}) - q^1_{j-1/2}(\bu^{n-1}) &= 0.
\end{align*}
\item %1
A backward-Euler time-discretization and a third-order, upwind-biased space-discretization:
\begin{align*}
q^1_{j+1/2}(\bu) - q^1_{j-1/2}(\bu)             &= v_0 \left( \tfrac{1}{3} u_{j-2} - \tfrac{6}{3} u_{j-1} + \tfrac{3}{3} u_j - \tfrac{2}{3} u_{j+1} \right), \\
q^1_{j+1/2}(\bu^{n-1}) - q^1_{j-1/2}(\bu^{n-1}) &= 0.
\end{align*}
\item %2
A trapezoid rule time-discretization and a centered (second-order) space-discretization.  Each of these lines approximates ``$\Delta x\,(v_0 u)_x$'':
\begin{align*}
q^1_{j+1/2}(\bu) - q^1_{j-1/2}(\bu)             &= v_0 \left( - \tfrac{1}{2} u_{j-1} + \tfrac{1}{2} u_{j+1} \right), \\
q^1_{j+1/2}(\bu^{n-1}) - q^1_{j-1/2}(\bu^{n-1}) &= v_0 \left( - \tfrac{1}{2} u_{j-1}^{n-1} + \tfrac{1}{2} u_{j+1}^{n-1} \right).
\end{align*}
\end{enumerate}

\medskip
In fact, the unified form in the code uses a vector of four coefficients $c_k$:
\begin{equation}
   q^1_{j+1/2}(\bu) - q^1_{j-1/2}(\bu) = v_0 \left( c_0 u_{j-2} + c_1 u_{j-1} + c_2 u_j + c_3 u_{j+1} \right). \label{eq:dfluxunified}
\end{equation}
Thus an array \verb|c[4]| of coefficients is set according to the option \verb|-lay_adscheme|.  Also, a PETSc stencil width of 2 is needed for multi-processor cases.


\section{Jacobian calculation}

We want to compute the nonzero derivatives
   $$\frac{\partial F_j(\bu)}{\partial u_k}$$
and insert them into a sparse matrix, the Jacobian; this is the $j$ row and $k$ column entry, in particular.  This section's computation occurs in method \verb|FormJacobianLocal()|, another call-back function, but it only happens if option \verb|-lay_jac| is set.  Otherwise the entries of the Jacobian are approximated within PETSc by calls to \verb|FormFunctionLocal()| and finite-differencing, which are made efficient by graph coloring.  There turns out to be a modest, but noticeable, benefit to having the analytical Jacobian computation we describe here.

Let
	$$S_1(u,z) = \frac{\partial}{\partial u} S(u,z) = - \gamma (n+2) u^{n+1} |z|^{n-1} z,$$
	$$S_2(u,z) = \frac{\partial}{\partial z} S(u,z) = - \gamma u^{n+2} (n) |z|^{n-1}.$$
The second formula uses the fact that the derivative of $\varphi(z) = |z|^{n-1} z$ on the whole real line is $\varphi'(z) = n |z|^{n-1}$.  Also define
\begin{align*}
S_k^{\text{left}} &= S_k\left(\frac{u_{j-1}+u_j}{2},\frac{u_j-u_{j-1}}{\Delta x} + b'_{j-1/2}\right), \\
S_k^{\text{right}} &= S_k\left(\frac{u_j+u_{j+1}}{2},\frac{u_{j+1}-u_j}{\Delta x} + b'_{j+1/2}\right),
\end{align*}
for $k=1,2$.

There are at most four nonzero entries per row $j$ of the Jacobian:
\begin{align*}
\frac{\partial F_j(\bu)}{\partial u_{j-2}} &= (1-\lambda) \nu_2 v_0 c_0 \\
\frac{\partial F_j(\bu)}{\partial u_{j-1}} &= \lambda \nu_2 \left( - S_1^{\text{left}} (\tfrac{1}{2}) - S_2^{\text{left}} (\tfrac{-1}{\Delta x}) \right) + (1-\lambda) \nu_2 v_0 c_1 \\
\frac{\partial F_j(\bu)}{\partial u_j} &= 1 + \lambda \nu_2 \left( S_1^{\text{right}} (\tfrac{1}{2}) + S_2^{\text{right}} (\tfrac{-1}{\Delta x}) - S_1^{\text{left}} (\tfrac{1}{2}) - S_2^{\text{left}} (\tfrac{1}{\Delta x}) \right) + (1-\lambda) \nu_2 v_0 c_2 \\
\frac{\partial F_j(\bu)}{\partial u_{j+1}} &= \lambda \nu_2 \left( S_1^{\text{right}} (\tfrac{1}{2}) + S_2^{\text{right}} (\tfrac{1}{\Delta x}) \right) + (1-\lambda) \nu_2 v_0 c_3
\end{align*}



%         References
%\bibliography{lc}
%\bibliographystyle{siam}




\end{document}
